%%%%%%%%%%%%%%%%%%%%%%%%% DO NOT CHANGE %%%%%%%%%%%%%%%%%%%%%%%%%%%
\documentclass[12pt]{article}
%%%%%%%%%%%%%%%%%%%%% END OF DO NOT CHANGE %%%%%%%%%%%%%%%%%%%%%%%

\usepackage[utf8]{inputenc}
\usepackage{setspace}
\usepackage{cite}




\setlength{\marginparwidth}{0pt}
\setlength{\marginparsep}{0pt} 
\setlength{\evensidemargin}{0.25in}  

\setlength{\oddsidemargin}{0.25in} 
\setlength{\textwidth}{6.375in}

%%%%%%%%%%%%%%%%%%%%% END OF DO NOT CHANGE %%%%%%%%%%%%%%%%%%%%%%%
\title{%
  Literature Review for Research Proposal \\
  \large The effects of remote work on employee productivity  \\
    BTM710 Assignment 3}




\author{Andy Mai, Md Iztiba, Phu Anh Pham, Bo Wei Yao, TJ LeBlanc}
\date{\today}


%%%%%%%%%%%%%%%%%%%%%%%%% DO NOT CHANGE %%%%%%%%%%%%%%%%%%%%%%%%%%%
\linespread{1.5}
%%%%%%%%%%%%%%%%%%%%% END OF DO NOT CHANGE %%%%%%%%%%%%%%%%%%%%%%%

\begin{document}
\maketitle

\section*{Research Question}
How does remote work affect employee productivity?

\section*{Hypotheses}
Null hypothesis: remote work has no effect on employee productivity. \\
Alternative hypothesis: remote work has either a positive or negative effect on employee productivity. 

\section*{Variables}
There are a number of variables that we are considering into measuring employee productivity: 

\subsection*{Major variables}
\begin{itemize}
  \item Self leadership 
  \item Mental health 
  \item Work environment 
  \item Time
\end{itemize}

\subsection*{Minor variables}
\begin{itemize}
  \item Training and career development 
  \item Physical health 
  \item Pay
  \item Tooling (communication, work essential tools)
  \item Process and Support (help desk, manager support)
\end{itemize}

\section*{Literature Review}

\subsection*{Self Leadership}

Working from home has enabled many people to work flexible hours and gain control of their lives. Many companies have now switched to a ��Results only Work Environment�� or ROWE, where employees can work from anywhere and for any amount of time as long as work is completed. This has shown to boost internal motivation due to the working environment being better and employees feeling a sense of self-leadership \cite{Abdul}. Companies have also realized that work from home is here to stay and need to adapt to the changing work environment. Thus, companies have started programs targeted at boosting self leadership and motivation levels for employees not in the office \cite{sultana2021exploring}. Studies have shown the best ways to do this are giving employees autonomy, a sense of power and flexible working hours \cite{sultana2021exploring}. 

On the other hand, companies that do not empower their work from home employees have shown to have worse work environments. Managers who make all the decisions have shown to have worse employee relationships, lower motivation and higher employee turnover. In fact, studies have shown that there is a 40.19\% correlation between worker satisfaction and productivity \cite{sultana2021exploring}. Satisfaction was seen to be higher when employees were given more trust by their managers to get the job done and not micromanaged even when working from home, thus leading to higher productivity \cite{sultana2021exploring}.

\subsection*{Health}

As the world comes out of the pandemic, we can learn more about the potential health benefits and negatives of remote work. Aside from the indirect benefits of remote work for mental health and stress from increased control of their time, there are also numerous health benefits as well \cite{doi:10.1177/1529100615593273}. Some of these include a significant reduction in work related stress and exhaustion as a direct result of having that control. There is also evidence to show the lack of travel due to remote work is actually a health benefit as well, as one study has shown that there is a negative correlation between those who commute and physical activity levels of those that do not\cite{HOEHNER2012571}. Another area of consideration has been the potential time saved from commuting and what that time could potentially be used for in terms of improving one's health. Studies have shown that people who do not commute to the office tend to eat less fast food and increase time spent at a gym \cite{allen2008workplace}.

However there are negative health considerations associated with the prevalence of remote work. The ergonomics of the working space has a major impact on health, which offices have standards and controls for but are not always present for remote work environments. Things to consider include back support, monitor height, arm rests and the location of keyboards and mice which can all contribute to detrimental work from home health and can increase risk of injury\cite{ellison2012ergonomics}.



\subsection*{Time}

Employees may minimize commute time by at least one and a half hours by remote working, allowing them to leave work early and have more time for their personal life \cite{george2022}. In addition, people experience less stress and are more productive as they can use the time saved from commuting to sleep more and feel better in the morning \cite{george2022}.

Other studies suggest that job satisfaction also increases when remote employment allows employees more autonomy, and time to satisfy their job, family and life \cite{natasha2016}. For instance, people can eat dinner with their families soon after their shift ends instead of spending more than thirty minutes traveling back home and feeling irritated by traffic congestion. Having high-quality time with family is vital to one��s emotions and mental health, and employees do more excellent quality work when they are in a pleasant mood and healthy mentality \cite{natasha2016}.

Also research on the effects of remote working during COVID-19 has shown that employees produce higher quality because they have longer breaks in between and can have flexibility in their shift as long as it is out of the meeting time \cite{wangb2021}.

Although remote work gives employees more personal time, there are some obstacles when working at home regarding the time. Employees may use the time saved from commuting to prepare meals (breakfast, lunch, dinner) for the whole family, which consumes considerable time and decreases their focus on work quality. In addition, employees spend more time assisting children and family members during the work shift and as a result, employees deal with more stress on the family conflict side, and their job productivity may be lower \cite{galanti2021}.



\subsection*{Work environment}

Working in a home environment can provide significant benefits over working in an office. One commonly seen problem with working in an office environment are distractions resulting from needing to constantly interact with co-workers. This is made much worse in offices that are open-plan, where noise and a lack of privacy also become very apparent concerns \cite{Kim2013}. Working from home helps solve these issues since employees would not need to deal with co-workers in person while also not being exposed to the generally chaotic nature of office environments. Employees are also given more flexibility to interact with their families due to not having to spend a large portion of their day away from home, which can contribute to a better work-life balance that improves both mental and physical health \cite{Xiao2021}. Additionally, it is essential for a work environment to have good lighting, ergonomics, air quality, acceptable temperature and humidity, and a low level of noise to ensure that a worker is satisfied and able to work at peak efficiency \cite{Xiao2021}. One of the benefits of working from home is that an employee has full control over these factors since it is their own home. This gives them the ability to customize aspects of their workspace as they wish, which can give them a great deal of satisfaction and comfort in comparison to being confined to a small cubicle \cite{Xiao2021}.

However, it is important to note that these benefits can greatly vary depending on many factors such as socioeconomic groups, occupations and industries that remote workers are in \cite{Etheridge2020}. For instance, in 2020 female workers and low earners generally reported suffering from a loss in productivity when they were forced to switch to remote work \cite{Abi2020}. This was primarily due to their occupations not being compatible with remote work \cite{AbiTasks2020}, and in the case of female workers distractions relating to childcare and other household chores also played a key role in the reduction of productivity \cite{Chattopadhyay2021}. As a result, these groups were found to have a notable decrease in their mental health due to stress \cite{Etheridge2020}. Stress resulting from social isolation can also become an issue that reduces productivity \cite{Toscano2020}, with this being made much worse for workers who suffer from psychological fragilities \cite{Bouziri2020}. Lastly, musculoskeletal disorders can appear in remote workers if the ergonomics of their workstations are unsatisfactory, resulting in a reduction in both productivity and job satisfaction \{Evina2020}.


\subsection*{Training and career development}

Events like COVID-19, followed by mandatory virtual or remote work settings, can have an impact on people's emotional and cognitive reactions, as well as their learning capacities and career development. COVID-19 has undoubtedly altered how human resources professionals and executives prepare people and organizations for change during uncertain times as well as how people react to the change. Emotional control affects how people process information and develop opinions, which can have significant effects on how they prepare for and make decisions about their careers \cite{Restubog2020}. To reduce the negative effects of emotions, one must actively and deliberately look for ways to manage them by participating in emotionally uplifting activities. Constant professional growth through virtual mentoring can also enable peak performance, regardless of one's career level. The chance for communication and professional growth are two benefits of virtual mentoring. It is organizations that will ultimately benefit from employees�� career development \cite{Yarberry2021}.  For more mutually beneficial outcomes, human resource development professionals are tasked to shift development initiatives to a virtual/remote environment in response to the new workplace normal.

Remote employees frequently experience social isolation since they do not have many face-to-face meetings and their communications with coworkers are irregular and limited \cite{Park2021}. They feel removed from decision-making processes and less connected to their organizations \cite{Virick2010}. Furthermore, in the context of distant e-work, the "out of sight, out of mind" approach might weaken the importance of personal connections. This might result in a person's career stagnation and professional growth, as well as limited access to social support systems including informal learning and mentorship \cite{Smith2018}.

Employees who were further along in their careers may have been shielded from the effects of missed opportunities for professional growth. This is demonstrated by the study's findings, which show that just seven percent of those over 55 say they would think about quitting their workplace if they did not receive more training and assistance. Instead, younger employees��a group most likely to gain from participating more fully in peer interactions and picking the brains of more experienced colleagues��have taken the brunt of missed chances. Furthermore, over half (46 percent) of workers under the age of 35 claim to have had less opportunities to interact with and learn from their coworkers during this time \cite{Kairinos2022}.

When it comes to becoming ready for the future of work, the workforce has seen significant advancements. Before COVID-19, technologies that are now progressively transforming into mainstays of the workplace setting were not even on employers' radars. With time, we feel optimistic that training leaders and HR managers will embrace more of these new and emerging technologies in a way that will enable the workforce to pursue individual career objectives and open doors to competitive learning opportunities \cite{Kairinos2022}.

Employers will be able to achieve potential advantages as well as keep a healthier, more productive staff by offering mindfulness training and coaching to individuals and teams \cite{Mariana}.



\bibliography{research_proposal}
%%%%%%%%%%%%%%%%%%%%%%%%% DO NOT CHANGE %%%%%%%%%%%%%%%%%%%%%%%%%%%

\bibliographystyle{ieeetr}
%%%%%%%%%%%%%%%%%%%%% END OF DO NOT CHANGE %%%%%%%%%%%%%%%%%%%%%%%

\end{document}
